% Flexibilidade e extensibilidade são alguns dos pricipais requisitos não funcionais desta arquitetura
% -----------------------------------------------------------
% Adicionar pelo menos 4;
% Adicionar uma tabela comparando as características do trabalho correlato com o meu;
% Ver como as tecnologias concorrentes como PhoneGap, Xamarin provêem apis de alto nível
% para atender aos requisitos não funcionais.
% Arquitetura de referência para mobile e Frameworks para mobile
% -----------------------------------------------------------

\section{TRABALHOS RELACIONADOS}
Nesta seção, serão apresentado os trabalhos relacionados com este projeto. Para cada trabalho relacionado, será descrito um resumo extraído do próprio artigo ou monografia e no final da seção, contém uma tabela que compara as principais características deste projeto com os trabalhos relacionados.\par

%artigo 1%
\textbf{Arquitetura de Referência para o Desenvolvimento de Sistemas Colaborativos Móveis Baseados em Componentes}\par
A arquitetura de referência proposta, denominada CReAMA – \textit{Component-Based Reference Architecture for Collaborative Mobile Applications}, teve como principal objetivo orientar o desenvolvimento de sistemas colaborativos móveis baseados em componentes para a plataforma Android. Sistemas desenvolvidos de acordo com essa arquitetura, devem dar suporte ao desenvolvimento de componentes e à criação de aplicações colaborativas por meio da composição desses componentes. As aplicações e componentes são desenvolvidos para plataformas móveis, facilitando o uso de recursos inerentes a essas plataformas, tais como informações de sensores embarcados. Com base na arquitetura de referência, o desenvolvedor poderá ser guiado para criar componentes e compor novas aplicações seguindo os padrões estabelecidos. Por exemplo, será possível construir \textit{toolkits} que forneçam componentes para um domínio específico. É importante ressaltar que a arquitetura foi definida considerando-se: aspectos da plataforma móvel, de sistemas colaborativos e da própria orientação a componentes. Com relação à plataforma móvel, optou-se por uma plataforma específica, visando-se a definição de uma arquitetura otimizada para as características da respectiva plataforma. A arquitetura proposta dará suporte ao desenvolvimento de novos sistemas baseados em componentes, considerando também aspectos relativos à comunicação com a Web.


%artigo 2%
\textbf{MoCA: Arquitetura para o Desenvolvimento de Aplicações Sensíveis ao Contexto para Dispositivos Móveis}\par
MoCA (\textit{Mobile Collaboration Architecture}) é uma arquitetura que oferece recursos para o desenvolvimento de aplicações distribuídas sensíveis ao contexto que envolvem usuários móveis. Esses recursos incluem um serviço para a coleta, armazenamento e distribuição de informações de contexto e um serviço de inferência de localização de dispositivos móveis. Além disso, a arquitetura provê APIs para o desenvolvimento de aplicações que interagem com estes servicos como consumidores de informações de contexto. Os servicos providos pela MoCA livram o programador da obrigação de implementar serviços específicos para a coleta e tratamento de contexto. O conjunto de APIs oferecidas pela MoCA para desenvolvimento de aplicações compreende tres grupos: as APIs de comunicação, que fornecem interfaces de comunicação sícrona e assícrona (baseada em eventos); as APIs principais que fornecem interfaces de comunicação com os serviços básicos da arquitetura; e as APIs opcionais que facilitam o desenvolvimento de aplicações baseadas na arquitetura cliente-servidor.\par


%artigo 3%
\textbf{Solução Multiplataforma para Smartphone Utilizando os Frameworks SenchaTouch e PhoneGap Integrado à Tecnologia WEB Service Java}\par
O trabalho teve como objetivo principal, realizar análise e estudo sobre as tecnologias de desenvolvimento de aplicativos móveis multiplataforma, utilizando a junção dos frameworks PhoneGap e Sencha Touch. Os aplicativos desenvolvidos usando o PhoneGap são aplicações híbridas onde partes do aplicativo, principalmente a interface do usuário, a lógica da aplicação e a comunicação com um servidor, é baseado em HTML, Javascript e CSS. A outra parte, que se comunica com o sistema operacional do dispositivo é baseada no idioma nativo de cada plataforma, ou seja, \textit{Java} no android e \textit{Objective C} no iOS. O estudo propôs uma modelagem facilitada de integração com outro sistema por meio de serviços web, através de uma aplicação RestFul utilizando Java EE. Com a análise das ferramentas e tecnologias levantadas, pode-se concluir que o desenvolvimento de aplicativos utilizando os frameworks PhoneGap e Sencha Touch tem muitas vantagens. Uma delas é a facilidade de portar o aplicativo para qualquer plataforma móvel. O  PhoneGap dispõe uma arquitetura MVC e diversos componentes para acesso a recursos do dispositivo, como câmera, acelerômetro e GPS através de objetos Javascript. O Sencha Touch dispõe de objetos focado em UI, principalmente suporte a eventos de toque na tela.\par


\begin{table*}[t]
\centering
\begin{tabular}{|c|c|c|c|c|c|}
\hline
Recurso & Este trabalho & CReAMA & MoCA & PhoneGap &  \\ \hline
\begin{tabular}[c]{@{}c@{}}Provê suporte multiPlataforma\\ mobile Android e iOS\end{tabular} & Sim & Não & Não & Sim &  \\ \hline
\begin{tabular}[c]{@{}c@{}}Provê suporte a\\ Plataforma Desktop\end{tabular} & Sim & Sim & Não & Não &  \\ \hline
\begin{tabular}[c]{@{}c@{}}Provê recursos extensíveis\\ através de plugins\end{tabular} & Sim & Não & Não & Não &  \\ \hline
\begin{tabular}[c]{@{}c@{}}Provê APIs de alto nível para recursos\\ de Rede (HTTP), banco de dados e UI\end{tabular} & Sim & Não & Sim & Sim &  \\ \hline
\begin{tabular}[c]{@{}c@{}}Provê suporte a\\ reuso de componentes\end{tabular} & Sim & Sim & Sim & Sim &  \\ \hline
\end{tabular}
\caption{Tabela comparativa dos recursos desta arquitetura com os trabalhos relacionados}
\label{my-label}
\end{table*}