% Flexibilidade e extensibilidade são alguns dos pricipais requisitos não funcionais da arquitetura

\section{TRABALHOS RELACIONADOS}
Nesta seção, serão detalhados os trabalhos relacionados com este projeto. O detalhamento será feito com uma descrição geral do trabalho desenvolvido e como ele se relaciona com este projeto. Será considerado também os pontos fracos e fortes identificados exibidos em uma simples tabela.

%artigo na pasta 1%
O trabalho apresentado por Maison Melotti \cite{melotti_2014} teve como objetivo definir uma arquitetura de referência denominada \textit{CREAMA} (\textit{Component-Based Reference Architecture for Collaborative Mobile Applications}) para aplicações colaborativas em dispositivos móveis Android baseadas em componentes. Melotti desenvolveu uma arquitetura com base nos requisitos de sistemas colaborativos, os requisitos foram levantados a partir de uma pesquisa sobre técnicas atuais de desenvolvimento de sistemas colaborativos e sistemas colaborativos móveis observando a partir do estado da arte, do suporte metodológico e computacional para o desenvolvimento dessa categoria de sistemas. Dentre as principais contribuições desse trabalho de Melotti, destaca-se: Mapeamento de diferentes requisitos genéricos ou frequentes de sistemas colaborativos móveis, elaboração de um cenário motivador para auxiliar na definição de requisitos de aplicações móveis baseadas em componentes e disposição de componentes para comunicação entre dispositivos e uso de sensores. É importante ressaltar que a arquitetura foi definida considerando-se: aspectos da plataforma móvel, de sistemas colaborativos e da própria orientação a componentes. Com relação à plataforma móvel, neste trabalho optou-se por uma plataforma específica, o Android. A tabela à seguir, apresenta um comparativo de atributos entre o projeto de arquitetura apresentado neste artigo e o trabalho de Melotti.

\begin{table}[H]
	\centering
	\label{my-label}
	\begin{tabular}{|l|c|c|}
		\hline
		\multicolumn{1}{|c|}{Recurso}                                                                 & \begin{tabular}[c]{@{}c@{}}Esta\\arquitetura \end{tabular} &
		\begin{tabular}[c]{@{}c@{}}Arquitetura\\de Melotti\end{tabular} \\ \hline
		\begin{tabular}[c]{@{}l@{}}Suporte multiplataforma\\(Android e iOS)\end{tabular} & Sim                                                                           & Não                                                                        \\ \hline
		\begin{tabular}[c]{@{}l@{}}Recursos extensíveis\\através de plugins\end{tabular}             & Sim                                                                           & Não                                                                        \\ \hline
		\begin{tabular}[c]{@{}l@{}}Provê componentes de\\alto nível para a aplicação\end{tabular}    & Sim                                                                           & Não                                                                        \\ \hline
		\begin{tabular}[c]{@{}l@{}}Suporte a comunicação\\através de eventos\end{tabular}            & Sim                                                                           & Não                                                                        \\ \hline
	\end{tabular}
	\caption{Comparação entre este trabalho e o de Melotti}
\end{table}

%artigo na pasta 2%
Estudo comparativo entre o desenvolvimento de aplicativos móveis utilizando abordagem nativa e multiplataforma


% Adicionar pelo menos 4;
% Adicionar uma tabela comparando as características do trabalho correlato com o meu;
% Ver como as tecnologias concorrentes como PhoneGap, Xamarin provêem apis de alto nível
% para atender aos requisitos não funcionais.
% Arquitetura de referência para mobile e Frameworks para mobile