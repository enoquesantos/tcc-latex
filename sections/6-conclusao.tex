\section{Conclusão}\label{sec:conclusao}
%\begin{enumerate}
%	\item retomar o texto, como na introdução e no resumo
%	\item reapresentar os resultados, conclusões...
%\end{enumerate}
% Aqui será apresentado as conclusões obtidos em todo o trabalho, sejam elas positivas ou negativas.
Este trabalho apresentou o projeto, implementação e documentação de uma arquitetura de software baseado em plugins para o desenvolvimento de sistemas de informação mobile. Destacou-se os recursos de baixo acoplamento entre objetos e facilidade de implementar features para comunicação com serviços \textit{RESTful}. A arquitetura destaca-se rica em recursos para operações básicas como acesso a rede, persistência de dados e notificações ao usuário tanto via push como local, através de APIs de alto nível. É possível ainda desenvolver objetos de UI usando componentes de alto nível através do QML.\par

Também foi realizado uma avaliação da arquitetura através do desenvolvimento de um aplicativo em duas versões, uma com e outra sem a arquitetura com o objetivo de analisar as vantagens e os benefícios de utilizar o framework proposto neste trabalho. Através da avaliação, concluiu-se que a arquitetura carece de depuração facilitada e melhorias na API de acesso a rede, além de uma documentação para cada uma das features disponibilizadas.\par

A característica de plugins da arquitetura desacopla as features do aplicativo do núcleo da aplicação, o que permite ao desenvolvedor adicionar, modificar ou remover recursos com maior facilidade, já que a comunicação entre objetos e recursos devem ser realizadas através de eventos. Através dos plugins, o desenvolvedor não precisa modificar o núcleo do aplicativo quando precisar alterar ou adicionar alguma funcionalidade, além de obter atualizações do framework sem perder as alterações feitas. Outra característica, é que o desenvolvedor irá escrever muito menos código, pois o framework já implementa muitas funcionalidades que os plugins podem utilizar através de APIs de alto nível, evitando ter que escrever código java, C++ ou Objective C. A arquitetura dispõe ainda de dois estilos para o aplicativo, usando \textit{containers} como \textit{StackView} e \textit{SwipeView} e permite ao desenvolvedor escolher qual usar editando apenas uma propriedade em um arquivo de configuração.

Os possíveis trabalhos futuros podem disponibilizar uma documentação das APIS da arquitetura, além de melhorar as features disponibilizadas. Os bugs identificados na avaliação também podem ser corrigidos e outros recursos podem ser adicionados no futuro a fim de incrementar funcionalidades através de APIs de alto nível a serem utilizadas pelos plugins. Dentre os trabalhos futuros, o suporte a plugins C++ é o maior dentre os desafios, pois permitiria aos plugins implementar features mais complexas através de implementações de mais baixo nível.

\subsection{Limitações Deste Trabalho}
A lista a seguir apresenta as limitações identificadas nesta arquitetura.

\begin{enumerate}
	\item Não há suporte a plugins escritos em C++: esse recurso seria importante, pois permitiria aos plugins delegar a lógica de negócio e operações de baixo nível a objetos C++, em vez de componentes QML como é atualmente. O problema é que as classes C++ devem ser conhecidas em tempo de compilação, pois em um projeto Qt as classes C++ devem estar mapeadas no arquivo \textit{.pro} gerando um acoplamento do núcleo da aplicação com os plugins;

	\item Instalação de plugins é feita somente durante o \textit{build:}: a arquitetura não suporta a instalação de plugins dinamicamente. Esse recurso permitiria estender as \textit{features} do aplicativo sem precisar de um \textit{rebuild} e novo \textit{deploy};

	\item Há suporte somente a um tipo de autenticação na API de rede: A arquitetura suporta apenas \textit{Basic Authentication} nas requisições HTTP e carece de outros tipos de autenticação suportados por \textit{web services} RESTful, tais como \textit{OAuth}, \textit{BEARER}, \textit{DIGEST Auth} entre outros;

	\item Atualização da arquitetura: Para criar um aplicativo utilizando este projeto requer alteração em diversos arquivos e isso implica em obter as atualizações futuras da arquitetura, pois os arquivos modificados serão sobrescritos e o desenvolvedor terá que atualizá-los manualmente. Seria interessante que todas as alterações necessárias fossem feitas em arquivos separados e durante o \textit{build} fosse feito o \textit{merge} entre os arquivos do núcleo e os arquivos modificados pelo desenvolvedor. Em projetos android usando a versão mais recente do gradle, é possível fazer \textit{merge} do arquivo \textit{AndroidManifest.xml} e o próprio arquivo \textit{build.gradle} que contém as APIs do \textit{Firebase} e outras propriedades de instalação e deploy no android.
\end{enumerate}


\subsection{Trabalhos Futuros}
Os itens abaixo, apresenta os possíveis incrementos futuros na arquitetura para facilitar ainda mais o desenvolvimento de plugins e melhorar a qualidade de um aplicativo baseado neste projeto.

\begin{enumerate}
	\item Adicionar os componentes reusáveis em um módulo estilo \textit{qmldir} para que o \textit{import} não seja feito através de uma string via \textit{qrc} e sim, via módulo similar ao \textit{import} do \textit{QtQuick};

	\item Adicionar o suporte a outros métodos na API de requisições HTTP tais como \textit{PUT}, \textit{OPTIONS}, \textit{DELETE} e \textit{HEAD};

	\item Atualizar a implementação de \textit{push notification} para a API em C++ do \textit{Firebase} que já está em versão estável. Isso irá simplificar a API de notificações via \textit{push} evitando ligações com objetos java via \textit{JNI};

	\item Melhorar o suporte ao iOS atualizando as bibliotecas do \textit{Firebase} para a versão mais recente, pois o \textit{xCode} exibe \textit{warnings} de que a versão da biblioteca utilizada possui APIs \textit{deprecated}, além de melhorar o suporte aos componentes visuais em telas retina declarando \textit{font-size} e dimensões baseadas no \textit{DPI} do dispositivo;

	\item Permitir que durante o desenvolvimento em modo Desktop, as alterações nos arquivos dos plugins sejam identificadas e carregadas a cada execução da aplicação, não exigindo o \textit{rebuild} do projeto, similar ao que acontece com os arquivos mapeados no \textit{qrc}. Atualmente, os plugins são copiados para o diretório do executável somente durante o \textit{build};

	\item Melhorar o suporte a depuração, com mensagens de erros e exceções mais detalhados nas APIs disponibilizadas (notificação, rede e persistência de dados).

	\item Desenvolver uma documentação com exemplos de código para cada \textit{feature} da arquitetura, com o objetivo de facilitar o desenvolvimento de plugins.
\end{enumerate}