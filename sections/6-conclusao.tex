\section{Conclusão}\label{sec:conclusao}
%\begin{enumerate}
%	\item retomar o texto, como na introdução e no resumo
%	\item reapresentar os resultados, conclusões...
%\end{enumerate}
Aqui será apresentado as conclusões obtidos em todo o trabalho, sejam elas positivas ou negativas.


\subsection{Limitações Deste Trabalho}
A lista a seguir apresenta as limitações identificadas nesta arquitetura. A implementação dessas limitações implicará em alterações diversas no núcleo da arquitetura, porém, não deve impactar nas implementações dos plugins. As limitações são decorrentes de pouca experiência com Qt, C++ e \textit{qmake}.

\begin{enumerate}
	\item Suporte a plugins escritos em C++: esse recurso seria importante, pois permitiria aos plugins delegar a lógica de negócio e operações de baixo nível a objetos c++, em vez de componentes qml como é atualmente. O problema é que, as classes C++ devem ser conhecidas em tempo de compilação, já que em um projeto Qt as classes C++ devem estar mapeadas no arquivo \textit{.pro} do projeto gerando um acoplamento do núcleo da aplicação com os plugins;

	\item Instalação de plugins em \textit{runtime}: a arquitetura não suporta a instalação de plugins dinamicamente. Atualmente, os plugins devem ser empacotados durante o \textit{build} do projeto;

	\item Tipos de autenticação na API de rede: A arquitetura suporta apenas \textit{Basic Authentication} nas requisições HTTP e carece de outros tipos de autenticação suportados por \textit{web services} RESTful, tais como \textit{OAuth}, \textit{BEARER}, \textit{DIGEST Auth} entre outros;

	\item Atualização da arquitetura: Para criar um aplicativo utilizando esta arquitetura requer alteração em diversos arquivos e isso implica em obter as atualizações futuras da arquitetura, pois os arquivos modificados serão sobrescritos e o desenvolvedor terá que atualizá-los manualmente. Seria interessante que todas as alterações necessárias fossem feitas em arquivos separados e durante o \textit{build} fosse feito o merge. Em projetos android usando a versão mais recente do gradle, é possível fazer merge do arquivo \textit{AndroidManifest.xml} e o próprio arquivo \textit{build.gradle} que contém as APIs do \textit{Firebase} e outras propriedades de instalação e deploy no android.
\end{enumerate}


\subsection{Trabalhos Futuros}
Os itens abaixo, apresenta os possíveis incrementos futuros na arquitetura para facilitar ainda mais o desenvolvimento de plugins e melhorar a qualidade de um aplicativo baseado neste projeto.

\begin{enumerate}
	\item Adicionar os componentes reusáveis em um módulo estilo \textit{qmldir} para que o \textit{import} não seja baseado em uma string via \textit{qrc};

	\item Adicionar o suporte a outros métodos na API de requisições HTTP tais como \textit{PUT}, \textit{OPTIONS}, \textit{DELETE} e \textit{HEAD};

	\item Atualizar a implementação de \textit{push notification} para a API em C++ do \textit{Firebase} que já está em versão estável. Isso irá simplificar a API de notificações via \textit{push} evitando ligações com objetos java via \textit{JNI};

	\item Melhorar o suporte ao iOS atualizando as bibliotecas do \textit{Firebase} para a versão mais recente, pois o \textit{xCode} exibe \textit{warnings} de que a versão da biblioteca utilizada possui APIs \textit{deprecated}, além de melhorar o suporte aos componentes visuais em telas retina declarando \textit{font-size} e dimensões baseadas no \textit{DPI} do dispositivo;

	\item Permitir que durante o desenvolvimento em modo Desktop, as alterações nos arquivos dos plugins sejam identificadas e carregadas a cada execução da aplicação, não exigindo o \textit{rebuild} do projeto, similar ao que acontece com os arquivos mapeados no \textit{qrc}. Atualmente, os plugins são copiados para o diretório do executável somente durante o \textit{build};

	\item Utilizar o \textit{Observer} como único mecanismo de comunicação entre objetos, pois o \textit{Observer} já está disponível na arquitetura mais não está sendo utilizado entre os componentes internos. As classes \textit{Subject} e \textit{Observer} em \textit{src/core} foram implementadas para melhorar ainda mais o mecanismo de comunicação por eventos reduzindo o \textit{overhead} na aplicação notificando somente os objetos interessados no evento, ao contrário do que ocorre atualmente.
\end{enumerate}