\documentclass[a4paper]{sig-alternate-05-2015}

%% regra para substituir as aspas convencionais para o padrão do latex :)
%% https://tex.stackexchange.com/questions/10670/quotes-in-latex
\newif\ifquoteopen
\catcode`\"=\active % lets you define `"` as a macro
\DeclareRobustCommand*{"}{%
	\ifquoteopen
	\quoteopenfalse ''%
	\else
	\quoteopentrue ``%
	\fi
}
% Deactive with: \catcode`\"=12\relax % changes `"` back to normal
\usepackage[utf8]{inputenc}
\usepackage[T1]{fontenc}
\renewcommand{\rmdefault}{phv} % Arial
\renewcommand{\sfdefault}{phv} % Arial
\usepackage{ae,aecompl}
\usepackage[portuguese]{babel}
\usepackage{hyphenat}
\usepackage{color}
\usepackage{float}
\usepackage{algorithm2e}
\usepackage{tabularx}
\usepackage{booktabs}
\usepackage{etoolbox}
\usepackage{graphicx}
\usepackage{pslatex}
\usepackage{blindtext}
\usepackage{hyperref}
\usepackage{caption}
\usepackage{tabularx}
\usepackage{ragged2e}
\graphicspath{{images/}, {diagramas/images/}}
\hyphenation{mate-mática recu-perar}

\begin{document}
	\makeatletter
	\def\fnum@figure{Figura \thefigure}
	\def\fnum@table{Tabela \thetable}
	\makeatother

	\title{Uma Arquitetura de Referência Baseada em Plugins para Sistemas de Informação Mobile}

	\numberofauthors{2}
	\author{
		% 1st. author
		\alignauthor
		Enoque Joseneas\titlenote{Graduando em Análise e Desenvolvimento de Sistemas}\\
		\affaddr{Instituto Federal da Bahia}\\
		\affaddr{Rua Emídio dos Santos, S/N, Barbalho}\\
		\affaddr{Salvador-Ba, Brasil}\\
		\email{enoquejoseneas@ifba.edu.br}
		% 2nd. author
		\alignauthor
		Sandro Andrade\titlenote{Prof. Doutor em Ciência da Computação}\\
		\affaddr{Instituto Federal da Bahia}\\
		\affaddr{Rua Emídio dos Santos, S/N, Barbalho}\\
		\affaddr{Salvador-Ba, Brasil}\\
		\email{sandroandrade@ifba.edu.br}
	}
	\maketitle

	\section*{RESUMO}
	O desenvolvimento de aplicativos móveis trouxe uma série de desafios para a ciência da computação. Com limitações de recursos como a bateria, armazenamento e memória, o desenvolvimento de software para dispositivos móveis impõe requisitos não-funcionais importantes a serem considerados no projeto de aplicativos. Com a popularização da internet e das redes móveis, os aplicativos móveis tornaram-se popular e projetar aplicativos de forma fácil com componentes de alto nível, baixo acoplamento e bom desempenho não é uma tarefa trivial. Este trabalho apresenta uma arquitetura de referência para o desenvolvimento de aplicativos móveis orientado a plugins no contexto de sistemas de informação, proporcionando baixo acoplamento entre os componentes, escalabilidade de \textit{features} através de plugins, dispõe de componentes genéricos reutilizáveis, além de componentes de alto nível para recursos corriqueiros como requisições HTTP, persistência de dados e comunicação entre os componentes através de eventos.

	\keywords{Aplicativos Móveis; Arquitetura de Software; Sistemas de Informação.}

	% sections
	% Outra justificativa:
% A atitude mais comum no desenvolvimento de uma nova aplicação tem sido a escolha de
% um framework de desenvolvimento disponível no mercado e de conhecimento da equipe que
% irá implementá-lo. Nesse ponto é que começam a aparecer os problemas, como forte
% acoplamento tecnológico, desperdício de esforços, dificuldade de gerenciamento de versão,
% uso exagerado e desnecessário de recursos computacionais, problemas estruturais, aumento do
% custo global, dificuldade de adequação a novas tecnologias, latência na geração de novas
% versões, etc. Todos esses problemas, naturais do processo de desenvolvimento de software,
% emperram a fabricação de software nas mais diversas áreas de aplicação.


\section{Introdução}\label{sec:introducao}
%--------------------------%
%Falar sobre a popularização dos dispositivos móveis e da Internet.%
%Falar sobre as mudanças introduzidas pelos dispositivos móveis nas formas de comunicação em sociedade.%
%--------------------------%
Os dispositivos móveis apresentam a cada dia novas oportunidades e desafios para as tecnologias de informação, tais como o acesso ubíquo, a portabilidade, a democratização do acesso à informação além de novas oportunidades de negócio \cite{levy_2002}. Com a expansão da Internet e o grande volume de dados compartilhados nas redes sociais e aplicativos de troca de mensagens, surgiram novos paradigmas (e.g. \textit{Big Data}, \textit{Cloud Computing}, \textit{NoSQL}, etc.), novas tecnologias como o \textit{Push Notification} e também novas oportunidades de trabalho e profissões (e.g. O analista de dados, o desenvolvedor mobile, o \textit{design UX} e etc.), além de pesquisas importantes na ciência da computação que abrange tanto hardware como software.\par

Os smartphones inovam a cada dia diversas áreas do conhecimento, tais como a engenharia elétrica, no projeto de baterias cada vez mais eficientes, o design, no projeto de interfaces cada vez mais intuitivas e influenciam diretamente na evolução da Internet e dos meios de comunicação como as redes móveis que expandem as áreas de cobertura para atender ao crescente número de aparelhos conectados. Os dispositivos móveis também permitem bons empreendimentos através dos aplicativos. Atualmente, o número de downloads cresce a cada dia na \textit{App Store} e \textit{Google Play}, demonstrando uma certa disponibilidade dos usuários de passarem cada vez mais tempo utilizando os aplicativos do que os próprios navegadores de Internet \cite{D&T}. Através dos aplicativos, é possível monetizar e gerar receitas via marketing digital e desenvolver soluções para diversos segmentos, tais como o \textit{e-commerce}, redes sociais e sistemas de informação.\par


%--------------------------%
%* Limitações das tecnologias atuais para desenvolvimento de sistemas client-servidor (RESTful) com dispositivos móveis:
%	1. falta de soluções arquiteturais de alto nível;
%	2. ausência de componentes de UI flexíveis e de alto nível;
%	3. falta de suporte nativo facilitado para comunicação RESTful;
%	4. push notification;
%	5. deploy de arquivos read-write;
%	6. operações disconectadas.
%--------------------------%
O desenvolvimento de aplicativos apesar de contar com inúmeras ferramentas tais como as IDEs (Android Studio, Eclipse e QtCreator) e frameworks (Ionic e PhoneGap), ainda apresentam limitações, dentre elas, a falta de soluções arquiteturais de alto nível, ausência de componentes de UI flexíveis e de alto nível. No Android por exemplo, para construir uma interface gráfica utiliza-se arquivos xml incorporados através de classes java. Outra limitação encontrada no desenvolvimento mobile, é a falta de suporte facilitado para comunicação RESTful, visto que os aplicativos móveis utilizam na maioria dos casos algum \textit{webservice}.


%--------------------------%
% Este trabalho teve como objetivo o projeto, implementação e avaliação de uma arquitetura flexível e reutilizável...%
%--------------------------%
Este trabalho teve como objetivo o projeto, implementação e avaliação de uma arquitetura orientada a plugins e reutilizável para o desenvolvimento de sistemas de informação mobile. Dentre os benefícios desenvolvidos destaca-se uma arquitetura de plugins, que permite ao desenvolvedor implementar as funcionalidades do sistema com maior facilidade de extensão e baixo acoplamento entre os componentes que podem se comunicar através de eventos. Esta arquitetura também provê componentes de alto nível para construção de interfaces gráficas através do qml, além de componentes de alto nível para operações rotineiras como ler e salvar dados no dispositivo através de um banco de dados SQLITE, realizar requisições HTTP com suporte a autenticação básica, download e upload de arquivos e também, permite o acesso aos arquivos do sistema que são compartilhados pelo usuário (galeria de arquivos e imagens). Esta arquitetura foi projetada a fim de atender aos seguintes requisitos funcionais: Acesso a rede através de comunicação HTTP com algum serviço REST (suporte a métodos GET, POST, upload e download de arquivos), persistência de dados local via SQLITE com tabelas definidas por cada plugin, notificações de sistema (local, partindo da própria aplicação quando estiver executando), \textit{push notification} através da API do Firebase\footnote{https://firebase.google.com}, suportando o registro do token e a exibição de notificações na bandeja do sistema.


%--------------------------%
%* Falar rapidamente de alguns aspectos técnicos: "implementado em Qt, bla, bla, bla ..."%
%--------------------------%
Para este trabalho foi utilizado como principal tecnologia o Qt, que provê um mecanismo de comunicação através de eventos, via sinais e slots e possibilita para a aplicação um meio de comunicação assíncrono entre objetos. O Qt também disponibiliza objetos de alto nível que abstrai a plataforma para operações de rede e persistência de dados. Outro recurso que o Qt provê é a instalação e a construção do arquivo executável da aplicação. O Qt também realiza o \textit{deploy} da aplicação em um dispositivo durante a fase de desenvolvimento, agilizando o processo de testes e correção de bugs do aplicativo.


%--------------------------%
%* Retórica *%
%--------------------------%
Este trabalho esta organizado como segue. A sessão 2 apresenta o referencial bibliográfico, destacando os assuntos emergentes relacionados a arquitura aqui apresentada. Em seguida, na sessão 3, será apresentado os trabalhos relacionados. As seções 4, 5 e 6 apresentam as tecnologias utilizadas e detalhes da solução desenvolvida. Por fim, na seção 7, é apresentado as perspectivas de trabalhos futuros e a conclusão deste trabalho.
	%Fazer um merge das duas últimas sessẽos para uma única mais simplificadamente.

%<apresentar uma pequena descrição dos assuntos apresentados nas sub-seções seguintes>%
%<Um ultimo paragrafo ligando estes quatro temas, sem ainda dizer o que você vai fazer>%
\section{REFERENCIAL BIBLIOGRÁFICO}
Esta seção, apresenta as principais referências que contextualizam este trabalho. A subseção 2.1 descreve sobre Arquitetura de Software. A subseção 2.2 descreve Visão Arquitetural. A subseção 2.3 resume Sistemas de Informação. Já a subseção 2.4 apresenta Arquitetura de Referência para Sistemas de Informação. A subseção 2.5 apresenta de forma geral Arquiteturas de Aplicativos Móveis. A subseção 2.6 faz uma breve revisão sobre Projetos de Aplicativos Móveis. A subseção 2.7 faz uma breve revisão sobre tecnologias suportadas na arquitetura deste trabalho: \textit{web services}, o estilo arquitetural \textit{REST}, o \textit{Push Notification} e o JSON. Para finalizar, a subseção 2.8 destaca sobre Desenvolvimento de Software Orientado a Componentes.


%<dizer os benefícios em trabalhar orientado a arquiteturas: controle intelectual, atendimento de requisitos não-funcionais, evolução, \textit{testability}, etc>%
%<dizer como as arquiteturas são projetadas>%
%<cada domínio de aplicação em particular demanda a adoção de arquiteturas particulares>%
\subsection{Arquitetura de Software}
De acordo com a definição clássica proposta por \textit{Shaw e Garlan} \cite{shaw_and_garlan}, arquitetura de software define o que é sistema em termos de componentes computacionais e os relacionamentos entre eles, os padrões que guiam suas composições e restrições. Arquitetura de software pode ser compreendida como uma especificação abstrata do funcionamento de um sistema e permite especificar, visualizar e documentar a estrutura e o funcionamento de um programa independente da linguagem de programação na qual ele será implementado \cite{jair_Cavalcanti_leite}.\par

Os softwares estão em constante evolução e sofrem mudanças periodicamente, que ocorrem por necessidade de corrigir \textit{bugs} ou de adicionar novas funcionalidades. As mudanças ocorridas no processo de evolução de um software podem torná-lo instável e predisposto a defeitos, além de causar atraso na entrega e custos acima do estimado. Porém, um software que é projetado orientado a arquitetura, possibilita os seguintes benefícios:
\begin{enumerate}
		\item Melhor escalabilidade;
		\item Maior controle intelectual;
		\item Menor impacto causado pelas mudanças;
		\item Melhor atendimento aos requisitos não-funcionais;
		\item Maior agilidade na manutenção do código;
		\item Padronização de comunicação entre os componentes e;
		\item Suporte a reuso dos componentes e maior controle dos mesmos.
\end{enumerate}

O desenvolvimento de software envolve muitas partes (e.g., levantamento de requisitos, modelagem, implementação, testes, refatoração e etc.). O objetivo de um software é o que motiva a sua construção, e o que fomenta todas as partes que envolve o seu desenvolvimento é o problema que ele tenta solucionar no mundo real e parte do mérito de uma boa solução é devido ao uso de uma boa arquitetura.

Neste trabalho, arquitetura de software pode ser compreendida nas decisões de implementação, nas restrições impostas pelo uso dos recursos disponibilizados e dos componentes reutilizáveis, além dos estilos arquiteturais provenientes das APIs utilizadas, tais como, o \textit{Event-Based}, mecanismo de comunicação orientado a eventos provido pelo Qt/QML e o \textit{Restful} que é utilizado como \textit{web service} suportado pela aplicação. Outro aspecto arquitetural deste trabalho é um estilo de desenvolvimento orientado a plugins que constituem os componentes específicos de cada projeto ou aplicação baseada nesta arquitetura. Os plugins formarão os principais recursos do sistema e são independentes entre si. O principal destaque em uma arquitetura de plugins é o baixo acoplamento entre as funcionalidades do sistema.


\subsection{Visão Arquitetural}
A arquitetura de um software pode ser representada de vários pontos de vista, que podem ser combinados para criar uma visão holística do sistema \cite{guideline:architectural_view}. As visões arquiteturais são diferentes formas de observar a arquitetura de um software, cada qual ressaltando aspectos específicos e relevantes conforme o papel da pessoa que está definindo a arquitetura e a etapa do processo de desenvolvimento em que ela se encontra \cite{Raymond1995}. Os requisitos funcionais implementados nesta arquitetura serão apresentados em um tópico seguinte em visões arquiteturais através de imagens e diagramas da UML. O objetivo das visões é facilitar a compreensão das partes que compõe esta arquitetura.


\subsection{Sistemas de Informação}
Um sistema de informação pode ser definido como um conjunto de componentes inter-relacionados trabalhando juntos para coletar, recuperar, processar, armazenar e distribuir informações com a finalidade de facilitar o planejamento, o controle, a coordenação, a análise e o processo decisório em organizações \cite{laudon}.


\subsection{Arquitetura de Referência para Sistemas de Informação}
Uma arquitetura de referência consiste em uma forma de apresentar um padrão genérico para um projeto \cite{zambiasi}. Com base nessa arquitetura, o desenvolvedor projeta, desenvolve e configura uma aplicação prototipando-a por meio de componentes reutilizáveis \cite{zambiasi}.\par

Para compor uma arquitetura de referência é necessário apresentar os tipos dos elementos envolvidos, como eles interagem e o mapeamento das funcionalidades para estes elementos \cite{Hofmeister:1999:ASA:322640}. De maneira geral, uma arquitetura de referência deve abordar os requisitos para o desenvolvimento de soluções, guiado pelo modelo de referência e por um estilo arquitetural de forma a atender as necessidades do projeto \cite{c._k_f._2006}. A concepção de uma arquitetura de referência pode ser entendida neste trabalho como uma forma de disponibilizar um padrão genérico para o desenvolvimento de novos aplicativos no contexto de sistemas de informação.\par

O domínio de aplicações móveis engloba vários requisitos e restrições que variam entre limitações de hardware tais como energia limitada, baixo poder de processamento e limitação de recursos como armazenamento e realizar comunicação remota quando o dispositivo está em rede móvel. Ao utilizar uma arquitetura de referência, é possível obter recursos implementados para funcionalidades corriqueiras no contexto da aplicação, como persistir dados, obter informações de um \textit{web service}, exibir uma notificação para o usuário e etc.


\subsection{Arquiteturas de Aplicativos Mobile}
Arquitetura para aplicações móveis abrange quatro camadas: Interação Humana-Computador (\textit{IHC}), Aplicação Móvel, \textit{Middleware} e \textit{Enterprise Backend} \cite{Pabllo:2008:MMA:1621087.1621128}. Neste trabalho, a arquitetura foi concentrada apenas nas camadas de interação, aplicação e \textit{Middleware}.

\subsubsection{Camada de Interação Humana-Computador}
A camada de Interação Humana-Computador (mais conhecida como interface de usuário, ou simplesmente UI) define os elementos de interação entre o usuário e os recursos do aplicativo. De forma abstrata, a camada de interface do usuário descreve o tipo de mídia suportada pelo aplicativo (por exemplo, texto, gráficos, imagens, vídeo ou som), os tipos de mecanismos de entrada (por exemplo, teclado alfa-numérico, ponteiros de caneta ou toques na tela) e os tipos de mecanismos de saída (por exemplo, uma notificação na bandeja do sistema, a tela, os alto-falantes ou algum tipo de \textit{feedback} como vibrar o dispositivo) \cite{Pabllo:2008:MMA:1621087.1621128}. Um exemplo de um componente desta camada é o objeto \textit{Image} do QML que corresponde ao carregamento e exibição de uma imagem na tela, além de botões, campos de texto e elementos que suportam cliques.

\subsubsection{Camada de aplicação}
A camada de aplicação corresponde ao processamento de ações e eventos provenientes da camada de interação, como por exemplo, captando eventos de toque em objetos visuais e realizando processamento em segundo plano como trocar a página atualmente vista pelo usuário, utilizando algum parâmetro lido em um objeto \textit{JSON}. Esta camada, corresponde a componentes não visuais e interagem diretamente com a camada de \textit{middleware}. Objetos da camada de aplicação podem por exemplo, gerenciar e controlar a criação de outros componentes tais como, o objeto \textit{StackView} do QML, que instancia páginas dinamicamente a partir de cliques em um menu de opções.

\subsubsection{Camada de middleware}
A camada de \textit{middleware} intercala entre a camada de aplicação com a camada de \textit{backend}. O objetivo dessa camada é fornecer de forma abstrata e genérica um meio de comunicação entre o modelo de dados da aplicação com a camada de \textit{backend} \cite{Pabllo:2008:MMA:1621087.1621128}. Ela é também responsável por interagir com o meio de comunicação disponível no dispositivo abstraindo para a camada de aplicação qual foi a interface de hardware utilizada. Objetos da camada de \textit{middleware} podem ser executados de forma assíncrona para que não bloqueiem os eventos da tela, enquanto aguardam um feedback da camada de \textit{backend} para permitir melhor desempenho e usabilidade da aplicação. Exemplos de componentes dessa camada são objetos que realizam acesso a rede através de requisições HTTP.

\subsubsection{Camada de backend}
A camada \textit{backend} consiste de uma outra aplicação que responde pelas requisições do aplicativo através de uma rede via protocolo \textit{HTTP}. Esta camada está associada ao \textit{web service} ou serviço REST. O \textit{web service} pode atender a diferentes requisições e dispositivos, além de abstrair para a aplicação, toda lógica de negócios referente ao armazenamento e processamento dos dados do sistema. A implementação desta camada pode ser desenvolvida sobre uma outra arquitetura, além de implementar regras de negócio inerentes ao seu funcionamento.

%caberia aqui uma imagem que mostrasse o relacionamento/interação entre essas 4 camadas!%



%<definir o que é um aplicativo móvel: de onde surgiu esse termo? Porque aplicativo e não aplicação ou mesmo software?>%
%<porque o uso tão popular dos aplicativos moveis?>%
%<características particulares dos aplicativos móveis.O que fazem eles diferentes das aplicações web, desktop, em cloud, etc?>%
%<demandas arquiteturais trazidas pelos aplicativos móveis>%
\subsection{Projeto de Aplicativos Móveis}
Um projeto é um esforço temporário empreendido para criar um produto, serviço ou resultado exclusivo. O termo temporário quer dizer que o projeto possui um ciclo de vida com início e final determinados \cite{governanadetidotcom}. O projeto termina quando seus objetivos forem alcançados ou quando existirem motivos para não continuá-lo \cite{governanadetidotcom}. Um aplicativo móvel ou aplicação móvel ou simplesmente \textit{app}, é um sistema desenvolvido para ser instalado e executado em um dispositivo eletrônico portátil, como tablets e smartphones \cite{what_is_mobile}.\par

Um aplicativo móvel pode ser baixado diretamente no aparelho eletrônico, desde que o dispositivo possua conexão com a Internet. O mercado de dispositivos móveis é ramificado por diferentes fabricantes, o que inclui uma variação de plataformas de desenvolvimento, sistemas operacionais, versões do SO e configuração variada de hardware. Na construção de um aplicativo para dispositivo móvel, a implementação é um ponto muito importante, pois, além de representar a parte concreta dos requisitos funcionais do aplicativo também refletem diretamente nos requisitos não funcionais e consequentemente na qualidade do software.\par

O sucesso de aplicativos para dispositivos móveis vai além das medidas de desempenho, portabilidade e usabilidade tradicionais \cite{Kronbauer:2012:UEE:2393536.2393582}. Os aplicativos devem estar em conformidade com a personalidade, preferências, objetivos, experiências e conhecimento de seus usuários \cite{Vermeeren:2010:UEE:1868914.1868973}. Além disso, o contexto físico, social e virtual onde ocorrem as interações deve, sempre que possível, ser levado em consideração \cite{McCarthy:2004:TE:1015530.1015549}.\par

Torna-se evidente que são muitos requisitos a serem considerados em um projeto de aplicativo móvel. O esforço dedicado para atender a todos os requisitos pode tornar o projeto enfadonho, além de exigir tempo e mão de obra. O processo de desenvolvimento pode ser otimizado através de ferramentas como \textit{frameworks} ou uma arquitetura de componentes reutilizáveis, a fim de agilizar e auxiliar o processo de desenvolvimento.


\subsection{Desenvolvimento de Software Orientado a Componentes}
O desenvolvimento de software Orientado a componentes é um paradigma da engenharia de software caracterizado pela composição de partes já existentes, ou desenvolvidas independentemente e que são integradas para atingir um objetivo final \cite{rafael_heider}. Construir novas soluções pela combinação de componentes desenvolvidos aumenta a qualidade e dá suporte ao rápido desenvolvimento, levando à diminuição do tempo de entrega do produto final ao mercado \cite{rafael_heider}. Os sistemas definidos através da composição de componentes permitem que sejam adicionadas, removidas e substituídas partes do sistema sem a necessidade de sua completa substituição. Com isso, o desenvolvimento baseado em componentes auxilia na manutenção do software, por permitir que o sistema seja atualizado através da integração de novos componentes ou atualização dos objetos já existentes \cite{szyperski_bosch_weck_1999}.\par

O reuso de componentes é um recurso extra da arquitetura apresentada neste trabalho, pois dispõe de trinta componentes (visuais e não visuais) reutilizáveis para auxiliar no desenvolvimento de novos aplicativos, seguindo as restrições desta arquitetura que é dedicada a sistemas de informação. Estes componentes são arquivos QML genéricos que podem atender a diferentes customizações através das propriedades disponibilizadas pelos seus elementos internos, que permitem definir ou alterar os valores pre-definidos. Os benefícios da componentização estão ligados a manutenibilidade, reúso, composição, extensibilidade, integração e escalabilidade \cite{D'Souza:1998:OCF:291139}.\par


%<introducao aos servicos web>%
%<tecnologias para servicos web: SOAP, RESTful, etc>%
%<definir o que é o RESTful>%
%<colocar uma figura explicando uma requisicao RESTful convencional>%
%<benefícios do RESTful: porque tem sido amplamente adotado?>%
\subsection{Web Services, RESTful, Push Notification e JSON}
Os \textit{web services} constituem uma tecnologia emergente da Arquitetura Orientada a Serviços (SOA) \cite{perepletchikov}. Com a expansão da internet e a necessidade de integração entre aplicações web, tornou-se necessário a centralização de informações para serem acessados por diferentes clientes. Para esse propósito, foi criada a tecnologia de \textit{web services} \cite{ibm_research}. Uma característica fundamental dos \textit{web services}, diz respeito à possibilidade de utilização de diferentes formas de transmissão de dados pela rede e o atendimento a diferentes clientes e dispositivos. Logo, a arquitetura de \textit{web services} pode trabalhar com protocolos HTTP, SMTP, FTP, RMI ou protocolos de mensagem proprietários.\par

o RESTFul é um estilo arquitetural para a construção de sistemas distribuídos \cite{fielding}. O elemento fundamental da arquitetura RESTful é o \textit{resource} ou recurso. Um recurso pode ser uma página web contendo um documento estruturado, uma imagem ou até mesmo um vídeo. Para localizar os recursos envolvidos em uma interação entre os componentes da arquitetura RESTful é utilizado o chamado identificador de recurso ou \textit{URI}. Com isso, um recurso pode ser representado através de diferentes formatos e o mais comum e utilizado é o \textit{JSON}. O termo REST, é empregado em serviços RESTful que não implementam todos os princípios da especificação do estilo arquitetural definido por \textit{Fielding} \cite{fielding}. Para que todos os princípios deste estilo sejam respeitados, um conjunto de restrições deve ser seguido, os quais não serão abordados neste trabalho. Para facilitar a compreensão geral de um serviço REST, a imagem à seguir foi adicionada para representar um modelo de comunicação entre um cliente ou dispositivo e um serviço RESTful.

\textit{Push Notification} é descrito por Acer et al. \cite{Acer:2015:EES:2902314.2902344} como mensagens pequenas, usadas por aplicações de celular para informar aos usuários sobre novos eventos e atualizações. As notificações na maioria dos casos, estão associadas aos aplicativos instalados no dispositivo. O termo \textit{push} indica que a mensagem parte do servidor para o dispositivo. Os principais provedores de notificações via \textit{push} são o \textit{Apple Push Notification Server} (APN) e o \textit{Firebase} antigo \textit{Google Cloud Messaging}. A imagem à seguir, apresenta o modelo de comunicação que ocorre no envio de um \textit{Push Notification}.\par

O JSON\footnote{https://json.org} (\textit{JavaScript Object Notation}) é um conjunto de chaves e valores, que podem
ser interpretados por qualquer linguagem. Além de ser um formato de troca de dados largamente utilizado em serviços REST, é fácil de ser entendido e escrito pelos programadores. Estas propriedades fazem do JSON um objeto ideal para o intercâmbio de dados em aplicações web tal como o XML \cite{jun_y_zhishu}.
	% Flexibilidade e extensibilidade são alguns dos pricipais requisitos não funcionais desta arquitetura
% -----------------------------------------------------------
% Adicionar pelo menos 4;
% Adicionar uma tabela comparando as características do trabalho correlato com o meu;
% Ver como as tecnologias concorrentes como PhoneGap, Xamarin provêem apis de alto nível
% para atender aos requisitos não funcionais.
% Arquitetura de referência para mobile e Frameworks para mobile
% -----------------------------------------------------------

\section{TRABALHOS RELACIONADOS}
Nesta seção, serão apresentado os trabalhos relacionados com este projeto. Para cada trabalho relacionado, será descrito um resumo extraído do próprio artigo ou monografia e no final da seção, contém uma tabela que compara as principais características deste projeto com os trabalhos relacionados.\par

%artigo 1%
\textbf{Arquitetura de Referência para o Desenvolvimento de Sistemas Colaborativos Móveis Baseados em Componentes}\par
A arquitetura de referência proposta, denominada CReAMA – \textit{Component-Based Reference Architecture for Collaborative Mobile Applications}, teve como principal objetivo orientar o desenvolvimento de sistemas colaborativos móveis baseados em componentes para a plataforma Android. Sistemas desenvolvidos de acordo com essa arquitetura, devem dar suporte ao desenvolvimento de componentes e à criação de aplicações colaborativas por meio da composição desses componentes. As aplicações e componentes são desenvolvidos para plataformas móveis, facilitando o uso de recursos inerentes a essas plataformas, tais como informações de sensores embarcados. Com base na arquitetura de referência, o desenvolvedor poderá ser guiado para criar componentes e compor novas aplicações seguindo os padrões estabelecidos. Por exemplo, será possível construir \textit{toolkits} que forneçam componentes para um domínio específico. É importante ressaltar que a arquitetura foi definida considerando-se: aspectos da plataforma móvel, de sistemas colaborativos e da própria orientação a componentes. Com relação à plataforma móvel, optou-se por uma plataforma específica, visando-se a definição de uma arquitetura otimizada para as características da respectiva plataforma. A arquitetura proposta dará suporte ao desenvolvimento de novos sistemas baseados em componentes, considerando também aspectos relativos à comunicação com a Web.


%artigo 2%
\textbf{MoCA: Arquitetura para o Desenvolvimento de Aplicações Sensíveis ao Contexto para Dispositivos Móveis}\par
MoCA (\textit{Mobile Collaboration Architecture}) é uma arquitetura que oferece recursos para o desenvolvimento de aplicações distribuídas sensíveis ao contexto que envolvem usuários móveis. Esses recursos incluem um serviço para a coleta, armazenamento e distribuição de informações de contexto e um serviço de inferência de localização de dispositivos móveis. Além disso, a arquitetura provê APIs para o desenvolvimento de aplicações que interagem com estes servicos como consumidores de informações de contexto. Os servicos providos pela MoCA livram o programador da obrigação de implementar serviços específicos para a coleta e tratamento de contexto. O conjunto de APIs oferecidas pela MoCA para desenvolvimento de aplicações compreende tres grupos: as APIs de comunicação, que fornecem interfaces de comunicação sícrona e assícrona (baseada em eventos); as APIs principais que fornecem interfaces de comunicação com os serviços básicos da arquitetura; e as APIs opcionais que facilitam o desenvolvimento de aplicações baseadas na arquitetura cliente-servidor.\par


%artigo 3%
\textbf{Solução Multiplataforma para Smartphone Utilizando os Frameworks SenchaTouch e PhoneGap Integrado à Tecnologia WEB Service Java}\par
O trabalho teve como objetivo principal, realizar análise e estudo sobre as tecnologias de desenvolvimento de aplicativos móveis multiplataforma, utilizando a junção dos frameworks PhoneGap e Sencha Touch. Os aplicativos desenvolvidos usando o PhoneGap são aplicações híbridas onde partes do aplicativo, principalmente a interface do usuário, a lógica da aplicação e a comunicação com um servidor, é baseado em HTML, Javascript e CSS. A outra parte, que se comunica com o sistema operacional do dispositivo é baseada no idioma nativo de cada plataforma, ou seja, \textit{Java} no android e \textit{Objective C} no iOS. O estudo propôs uma modelagem facilitada de integração com outro sistema por meio de serviços web, através de uma aplicação RestFul utilizando Java EE. Com a análise das ferramentas e tecnologias levantadas, pode-se concluir que o desenvolvimento de aplicativos utilizando os frameworks PhoneGap e Sencha Touch tem muitas vantagens. Uma delas é a facilidade de portar o aplicativo para qualquer plataforma móvel. O  PhoneGap dispõe uma arquitetura MVC e diversos componentes para acesso a recursos do dispositivo, como câmera, acelerômetro e GPS através de objetos Javascript. O Sencha Touch dispõe de objetos focado em UI, principalmente suporte a eventos de toque na tela.\par


\begin{table*}[t]
\centering
\begin{tabular}{|c|c|c|c|c|c|}
\hline
Recurso & Este trabalho & CReAMA & MoCA & PhoneGap &  \\ \hline
\begin{tabular}[c]{@{}c@{}}Provê suporte multiPlataforma\\ mobile Android e iOS\end{tabular} & Sim & Não & Não & Sim &  \\ \hline
\begin{tabular}[c]{@{}c@{}}Provê suporte a\\ Plataforma Desktop\end{tabular} & Sim & Sim & Não & Não &  \\ \hline
\begin{tabular}[c]{@{}c@{}}Provê recursos extensíveis\\ através de plugins\end{tabular} & Sim & Não & Não & Não &  \\ \hline
\begin{tabular}[c]{@{}c@{}}Provê APIs de alto nível para recursos\\ de Rede (HTTP), banco de dados e UI\end{tabular} & Sim & Não & Sim & Sim &  \\ \hline
\begin{tabular}[c]{@{}c@{}}Provê suporte a\\ reuso de componentes\end{tabular} & Sim & Sim & Sim & Sim &  \\ \hline
\end{tabular}
\caption{Tabela comparativa dos recursos desta arquitetura com os trabalhos relacionados}
\label{my-label}
\end{table*}
	\section{Projeto da arquitetura}
A arquitetura proposta neste trabalho é baseada em camadas e utiliza os estilos arquiteturais \textit{Client-Server} e \textit{Event-Based}. O modelo em camadas possibilita manter a organização, a separação de conceitos e responsabilidades dos recursos da arquitetura, viabilizando a integração e comunicação entre os componentes através de conectores que podem ser definidos dinamicamente. No modelo em camadas, a conexão entre os componentes pode ser realizado tanto por eventos como por objetos compartilhados ou, através de leitura e escrita em um arquivo ou em uma base de dados. Porém, nesta arquitetura, foi utilizado somente eventos para comunicação entre os objetos da aplicação. A escolha de eventos como principal conector entre os objetos foi feita principalmente por proporcionar comunicação assíncrona entre emissor e ouvinte e pelo fato de não acoplar os componentes, garantindo maior independência entre os objetos e permitir que os plugins possam transmitir dados para outros objetos ou até mesmo outros plugins.\par


\subsection{Processos de desenvolvimento}
Esta arquitetura foi desenvolvida sob uma metodologia ágil com destaque para uma programação extrema e teste contínuo. A arquitetura recebeu alterações durante 10 meses e a primeira etapa de desenvolvimento introduziu o suporte aos plugins. O primeiro desafio foi desacoplar os plugins do arquivo QRC\footnote{QRC - Qt Resource Collection é um arquivo XML que mapeia os arquivos que serão empacotados no aplicativo} e permitir que a aplicação carregasse-os dinamicamente. Também nesta primeira etapa, foi implementado alguns recursos associados aos plugins, como controle de cache dos arquivos QML, ordenação e \textit{parsing} das páginas (definido pelos plugins), além da criação de um componente genérico a ser implementado por todas as páginas do aplicativo. O controle de cache consiste em regenerar o cache de todos os arquivos da aplicação após uma atualização e se faz necessário para garantir o carregamento de mudanças em cada arquivo a cada release. O componente genérico foi definido como \textit{BasePage.qml} e será detalhado posteriormente, ele foi criado para garantir o atendimento de alguns requisitos mínimos de aparência e estrutura da aplicação além de simplificar a criação de páginas. Na segunda etapa, foi implementado uma classe utilitária para que seus métodos fossem utilizados pelos plugins, oferecendo operações de baixo nível ainda não suportados pelo QML, pois nesta versão da arquitetura, ainda não é suportado a implementação de objetos c++ por conta dos plugins. Na terceira etapa, foi definido os layouts visuais suportados pela arquitetura e dois modelos foram implementados: O layout em pilha (faz uso do container \textit{StackView}) e o layout em linha (faz uso do container \textit{SwipeView}). O suporte ao layout deve ser definido no arquivo de configuração principal (à ser detalhado posteriormente), outros componentes serão instanciados para uso em conjunto com cada layout, dentre eles o TabBar e o ToolBar. Em etapas seguintes foi desenvolvido componentes visuais reutilizados na aplicação além das APIs para requisição HTTP e persistência de dados via QSLITE.


\subsection{Tecnologias utilizadas}
As tecnologias utilizadas consiste de todos os recursos que foram necessários para o desenvolvimento deste trabalho. O Qt e o \textit{QtCreator} foram os artefatos mais importantes, pois, forneceram os recursos e ferramentas para a construção das principais características da arquitetura. Dentre os recursos providos pelo Qt destaca-se os eventos, que permitem interligar objetos através de sinais e slots\footnote{funções javascript ou métodos de uma classe c++ invocados quando o sinal o qual estão conectados for emitido, recebendo em seus parâmetros os argumentos enviado pelo sinal.} ou \textit{signal handles}, e as APIs providas em classes C++ que integram os recursos da arquitetura, tais como, persistência de dados (via \textit{QSettings} e \textit{QSqlDatabase}) e rede (via \textit{QNetworkAccessManager}). O \textit{QtCreator} é uma IDE que possui recursos integrados à um projeto Qt e foi essencial para o desenvolvimento deste trabalho. Alguns recursos do \textit{QtCreator} destaca-se, facilidade de \textit{build} do projeto, construção do executável do aplicativo e o \textit{deploy} em um \textit{smartphone}, além de facilitar a realização de testes através de um mecanismo integrado de depuração. O \textit{QtCreator} também foi utilizado como editor de código fonte de todo o projeto da arquitetura.


% ver link: https://pt.slideshare.net/adrianotavares/modelagem-arquitetural-e-viso-41-presentation %
\subsection{Visões da Arquitetura}
As visões a seguir, apresentam a arquitetura através de diagramas e destacam diferentes pontos de vista. O objetivo das visões é permitir compreender melhor a arquitetura e as principais partes que a compõe. Porém, não será apresentado nenhum dos elementos, módulos e APIs do Qt, pois é desenvolvido sob outras arquiteturas que vão muito além do escopo deste trabalho.\par
\begin{figure}[h]
	\includegraphics[scale=0.5]{diagrama_geral_da_arquitetura}
	\centering
	\caption{Pacotes principais da arquitetura}
\end{figure}
A figura 3, apresenta uma visão geral dos principais pacotes e arquivos da arquitetura. A seguir, é descrito o papel e o conteúdo de cada um dos arquivos e pacotes.
\begin{description}
	\item[1] \textit{tcc.pro}: Arquivo de configuração de todo projeto Qt. Nele é definido os módulos do Qt a serem utilizados na aplicação, as classes c++ que serão compiladas e linkadas no executável, os aquivos de \textit{resources} (arquivos onde estão mapeados os componentes QML, as imagens e etc.), além de módulos e arquivos de configuração para cada plataforma (desktop, osx, android e ios) e o diretório onde os plugins serão instalados no dispositivo.
	\item[2] \textit{main.cpp}: Ponto de entrada da aplicação. Responsável por instanciar as classes do Qt que exibem a janela do aplicativo e o interpretador de QML, além de classes de configuração e utilitários. Também é responsável por carregar os arquivos de tradução e registrar objetos no contexto da aplicação para uso pelos plugins;
	\item[3] \textit{config.json}: Arquivo de configuração principal onde deve ser adicionado as definições de layout a ser usado pelo aplicativo, por padrão usará layout em pilha. Outra propriedade que deve ser definido é o nome e a descrição do aplicativo, além do nome da organização e o domínio (ambos utilizados pelo Qt para criar e identificar as configurações no dispositivo na home do usuário). É também neste arquivo onde dever ser definido as cores utilizadas pelos plugins. As cores também serão utilizadas em componentes internos, evitando manter a cor fixa sendo definida dinamicamente em componentes visuais.
	\item[4] \textit{src}: Diretório de código fonte. É onde está as classes c++ e componentes QML usados internamente e dispostas pelos plugins como componentes reusáveis. Esse diretório é sub-dividido em outros 5 diretórios detalhados em outro diagrama. As classes contidas nesse diretório são mapeadas no arquivo \textit{tcc.pro}.
	\item[5] \textit{plugins}: Diretório de plugins. Cada plugin deve obrigatóriamente estar em um sub-diretório com no mínimo um arquivo de configuração de nome \textit{config.json} e os arquivos QML e imagens que necessitar para o seu funcionamento. Os detalhes das propriedades requeridas para carregamento do plugin serão descritas em um tópico posterior.
	\item[6] \textit{translations}: Diretório contendo os arquivos de tradução. Os arquivos de tradução devem ser gerados antes de cada release do aplicativo caso haja alterações nas strings exibidas para o usuário. Um arquivo de resources \textit{translations.qrc} existe neste diretório e deve ser utilizado para mapear os idiomas suportados pelo aplicativo. Cada arquivo de tradução deve ser nomeado seguindo o padrão idioma_PAÌS com a extensão \textit{.ts}, por exemplo: pt_BR.ts. Ao iniciar a aplicação, no arquivo main.cpp, será identificado o \textit{locale} que define o idioma utilizado no dispositivo e o arquivo correspondente será instanciado e as strings serão traduzidas seguindo a disposição definida no arquivo traduzido. Para regenerar as traduções, deve-se utilizar o comando \textit{lupdate *.pro} para criar o atualizar o arquivo \textit{.ts}.
	\itrem[7] \textit{android}: Diretório contendo os arquivos de configuração do aplicativo para a plataforma android. Outros sub-diretórios guardam arquivos do gradle utilizados para o build do APK, ícones do lançador do aplicativo e classes java, além de uma versão da lib \textit{openssl} compilada para o funcionamento de requisições HTTP.
	\item[8] \textit{assets}: Diretório contendo imagens e arquivos de configuração do \textit{qtquickcontrols2} além de um arquivo html que pode ser usado para exibir os termos de uso do aplicativo caso necessário. Um arquivo de \textit{resources} \textit{assets.qrc} mapeia todos os arquivos contidos neste diretório e pode ser usado para empacotar outros componentes e imagens do usuário no aplicativo.
	\item[9] \textit{ios}: Diretório contendo os arquivos de configuração do aplicativo para a plataforma ios. 
\end{description}

\subsection{Métodos para utilização da arquitetura}
Exemplo (a ser editado): Para se utilizar a arquitetura desenvolvida, deve-se seguir uma determinada ordem de atividades (Figura 26), que deve se iniciar no nível arquitetural “escopo” (seção 3.1), passando pelos níveis arquiteturais “modelo de negócios” (seção 3.2) e “modelo de sistema” (seção 3.3), até chegar ao nível arquitetural “modelo tecnológico” que deve ser criado pelo usuário desta arquitetura.


\subsubsection{Visão lógica}
A imagem à seguir, apresenta uma visão lógica do projeto da arquitetura através do diagrama de componentes da UML. O objetivo dessa visão, é descrever os principais módulos do sistema e como eles estão relacionados partindo do modelo de arquiteturas de sistemas \textit{mobile}.


\subsection{Modelagem da arquitetura de plugins}
A modelagem de plugins marcou o início deste trabalho e desacoplá-los do núcleo da aplicação foi uma terafa importante, mantendo-os independentes entr si, permitindo desta forma, que novos arquivos e plugins pudessem ser adicionados ou removidos sem impactar ou ter que alterar o núcleo do aplicativo. Com isso, o núcleo se tornaria independente dos plugins, e os plugins passariam a ser conhecidos dinamicamente. A partir da arquitetura de plugins, os requisitos foram levantados seguindo as \textit{features} providas pelo Emile, onde se fez necessário implementações para atender as demandas dos requisitos funcionais do aplicativo, tais como, uma API de persistência de daos usado no perfil do usuário (um objeto com diferentes propriedades) e a criação de tabelas em um banco sqlite disponíveis pelos plugins. Outro recurso foi uma API para acesso a rede (requisições http para comunicação com o serviço REST). 

Para gerenciar os plugins, foi criado uma classe c++ que realiza o processo de busca, ordenação e registro dos arquivos dos plugins a serem usados na aplicação. Também foi definido algumas restrições de uso para os plugins e será discutido em sessão posterior.


%adicionar um diagrama de componentes mostrando os plugins em um diretório de plugins e outros diretórios da aplicação
\subsubsection{O Objeto PluginManager}\label{sec:solucao-desenvolvida}
O Objeto PluginManager....


%adicionar um diagrama de componentes mostrando os plugins em um diretório de plugins e outros diretórios da aplicação
\subsubsection{O Qt e o QML}\label{sec:solucao-desenvolvida}
Para compreender a implementação deste trabalho é importante entender os recursos do Qt e como ele facilitou o desenvolvimento desta arquitetura. O Qt é a base da implementação deste trabalho, ele provê um conjunto de bibliotecas, classes e componentes além de uma infraestrutura de desenvolvimento, depuração e deploy. O Qt também fornece APIs de alto nível que abstrai a plataforma e o sistema operacional e permite integrar objetos através de um mecanismo de comunicação orientado a eventos. O Qt consiste de um kit de ferramentas composto de classes C++, um interpretador de linguagem declarativa chamada QML além de Módulos com APIs de rede, persistência de dados, recursos de hardware como câmera, bluetooth, sensores e etc. Além de uma ampla de componentes visuais de alto nível que facilitam a construção de interface gráfica. Os elementos seguintes são áreas-chave de funcionalidades do Qt \cite{qt_overviews}.

\begin{itemize}
	\item Ferramentas de Desenvolvimento através da IDE Qt Creator que fornece depuração e deploy simplificado
	\item Armazenamento de dados em alto nível de abstração e independente de plataforma
	\item Suporte a rede (http) e conectividade também independente de plataforma
	\item Gráficos proeminentes providos pelo QML através do módulo \textit{Qt Quick} e \textit{Quick Controls}
	\item Portabilidade com alta abstração de plataforma para elementos visuais como imagens, botões e exibição de textos
	\item Desenvolvimento aberto através da licença open source
\end{itemize}

\subsection{Comunicação entre os componentes}
Esta seção descreve o processo de comunicação entre objetos ou componentes dentro do aplicativo...


\section{Aspectos de Implementação}\label{sec:solucao-desenvolvida}
Nesta sessão é discutido detalhes da solução desenvolvida. Será apresentado nas subseções seguintes os componentes que integra a arquitetura desenvolvida, descrevendo as decisões e características de implementação de cada componente.



\subsection{O Objeto App}\label{sec:solucao-desenvolvida}
O objeto app é um componente importante nesta arquitetura, ele gerencia toda a persistência de dados através de uma instância da classe \textit{QSettings} que abstrai a plataforma e fornece um mecanismo para leitura e escrita de dados no dispositivo através de métodos parametrizados. O objeto app é também o responsável por instanciar a classe \textit{PluginManager}, que realiza o trabalho de carregar os plugins do aplicativo. Outra responsabilidade deste objeto, é a notificação de eventos vindos do objeto \textit{QtActivity} no android e \textit{QtAppDelegate} no iOS como o \textit{token} de registro no \textit{Firebase}\footnote{O Firebase antigo \textit{Google Cloud Messaging} é um serviço de notificação móvel desenvolvido pelo Google que permite que os desenvolvedores de aplicativos enviem dados ou informações de notificação de servidores executados pelo desenvolvedor para aplicativos Android e iOS.} e mensagens de \textit{push notification} que em ambos os eventos, são delegados para o aplicativo através de uma conexão com um \textit{slot} definido no \textit{Application Window}.

\begin{figure}[h]
	\includegraphics[width=8cm]{a_classe_app}
	\centering
	\caption{Os objetos compostos da classe App}
\end{figure}

O Objeto App também fornece métodos públicos declarados como 

\subsection{O arquivo config.json}\label{sec:solucao-desenvolvida}
O arquivo \textit{Settings.json} é um arquivo importante e requerido nesta arquitetura. As propriedades presentes neste arquivo não são persistidas no banco de dados e serão setadas em um objeto do tipo \textit{QVariantMap} na classe App, sempre que o aplicativo for executado. Este objeto será registrado no contexto da aplicação como uma propriedade e será utilizado por vários componentes, tais como, elementos visuais que definem cores, font-size e outras propriedades que podem ser definidas neste json e poderão ser acessadas por qualquer plugin. Algumas propriedades como versão do aplicativo e se está ou não em modo de depuração. O número da versão é especificado na chave \textit{version} e é utilizado pelo objeto ManagerPlugins que decidirá se o cache dos arquivos QML serão deletados para que os arquivos QML presentes na nova versão sejam regenerados. será utilizado por mapear tanto o tema de cores utilizado pelos componentes visuais como também detalhes de nome, versão e descrição do aplicativo.

Tem que ter uma visão estrutural do núcleo da 

\subsection{O Componente RequestHttp}\label{sec:solucao-desenvolvida}
O Objeto RequestHttp....

\subsection{O Componente Basepage}\label{sec:solucao-desenvolvida}
O Objeto ModelData....

Esta seção deve apresentar: 
\begin{enumerate}
\item A solução desenvolvida, se isto já for o trabalho final. Neste caso o tempo verbal usado é passado. Foi desenvolvido, foi elaborado...
\item A proposta de solução, se for o pré-projeto. Neste caso o tempo verbal é futuro. Será desenvolvido, será elaborado...
\end{enumerate}
Uma boa forma de começar esta seção:
Nesta seção é discutido a solução proposta/desenvolvida.
%A Figura \ref{fig:visao-geral} apresenta uma visão geral da solução.
%Explique cada um dos elementos da figura. Use uma notação correta.
%Exemplo de algoritmo \ref{alg:BA}.

	%\section{Avaliação Experimental}\label{sec:avaliacao-experimental}
Aqui será apresentado a estudo de avaliação conduzido para valiar a solução desenvolvida.
	%\section{Conclusão}\label{sec:conclusao}
%\begin{enumerate}
%	\item retomar o texto, como na introdução e no resumo
%	\item reapresentar os resultados, conclusões...
%\end{enumerate}
Lorem ip sum....\par
Lorem ip sum....\par
Lorem ip sum....\par


\subsection{Limitações Deste Trabalho}
A lista a seguir, apresenta as limitações identificadas nesta arquitetura. A implementação dessas limitações implicará em alterações diversas no núcleo da arquitetura, porém, não impactaria no funcionamento dos plugis atualmente. No entanto, as limitações são decorrentes de pouca experiência com qmake e Qt.

\begin{enumerate}
	\item Suporte a plugins escritos em c++: esse recurso seria importante, pois permitiria aos plugins delegar a lógica de negócio e operações de baixo nível a objetos c++, em vez de componentes qml como é atualmente. O problema é que, as classes c++ devem ser conhecidas em tempo de compilação, já que as classes c++ de um projeto Qt devem estar declaras no arquivo \textit{.pro} do projeto. Neste caso, o núcleo da arquitetura teria que ser modificado pelo desenvolvedor, pois os objetos c++ teriam ser conhecidos por objetos qml em tempo de execução e isso requer uma refatoração de partes da arquitetura;

	\item Instalação de plugins em \textit{runtime}: a arquitetura não suporta ainda a instalação de plugins dinamicamente. Atualmetne, os plugins de um aplicativo devem ser empacotados durante o \textit{build} do projeto;

	\item A arquitetura suporta apenas \textit{Basic Authentication} nas requisições HTTP e carece de outros tipos de autenticação suportado por \textit{web services} RESTful, tais como \textit{OAuth}, \textit{BEARER}, \textit{DIGEST Auth} entre outros;
\end{enumerate}


\subsection{Trabalhos Futuros}
A lista a seguir, apresenta os possíveis incrementos futuros na arquitetura para facilitar ainda mais o desenvolvimento de plugins pelos desenvolvedores e melhorar a qualidade de um aplicativo baseado neste projeto.

\begin{enumerate}
	\item Melhorar o suporte ao iOS em componentes visuais melhorando o \textit{look end feel} nesta plataforma;

	\item Permitir que durante o desenvolvimento, quando executando em modo Desktop, as alterações nos arquivos de plugins sejam identificados a cada execução da aplicação, não exigindo o \textit{rebuild} do projeto;

	\item Adicionar um modelo de comunicação estilo \textit{publish-subscriber} migrando a comunicação por eventos para o \textit{Observer} que já está disponível na arquitetura mais não está sendo utilizado. As classes \textit{Subject} e \textit{Observer} em \textit{src/core} foram implementadas para melhorar ainda mais o mecanismo de comunicação por eventos
\end{enumerate}
	%%\input{sections/apendice-a.tex}

	% to produce the bibliography for the citations in your paper.
	\bibliographystyle{abbrv}
	\bibliography{references}

\end{document}